\documentclass[10pt,twocolumn,letterpaper]{article}

\usepackage{cvpr}
\usepackage{times}
\usepackage{epsfig}
\usepackage{graphicx}
\usepackage{amsmath}
\usepackage{amssymb}

% Include other packages here, before hyperref.

% If you comment hyperref and then uncomment it, you should delete
% egpaper.aux before re-running latex.  (Or just hit 'q' on the first latex
% run, let it finish, and you should be clear).
\usepackage[pagebackref=true,breaklinks=true,letterpaper=true,colorlinks,bookmarks=false]{hyperref}

\cvprfinalcopy % *** Uncomment this line for the final submission

\def\cvprPaperID{****} % *** Enter the CVPR Paper ID here
\def\httilde{\mbox{\tt\raisebox{-.5ex}{\symbol{126}}}}

% Pages are numbered in submission mode, and unnumbered in camera-ready
\ifcvprfinal\pagestyle{empty}\fi
\begin{document}

%%%%%%%%% TITLE
\title{Using RRTs to Navigate Gibson Environments}

\author{Paul McKerley\\
George Mason University\\
{\tt\small pmckerle@masonlive.gmu.edu}
}

\frenchspacing
\maketitle
\thispagestyle{empty}

%%%%%%%%% ABSTRACT
\begin{abstract}
  Produce efficient maps to navigate two-dimensional floorplans
  extracted from three-dimensional Gibson Environment buildings.
\end{abstract}

%%%%%%%%% BODY TEXT
\section{Introduction}

  The goal of this project was to produce road maps for Gibson
  Environment buildings that would allow robots to navigate from one
  arbitrary point in the map to another. This navigation should find
  reasonable straight-line paths. My approach was to use
  Rapidly-Exploring Random Trees, which provided maps in a matter of
  seconds. A* search can traverse these maps, and the results paths
  refined to be straighter. Many samples paths can be found and
  refined, with the resulting paths added to the map. These refined
  paths give highly efficient paths between rooms.

\section{Approach}

There are several steps requirement to produce a roadmap of a Gibson
Environment structure.

\subsection{Extract Two-Dimensional Floorplan}

The Gibson buildings come as three-dimensional graphs. To extract
two-dimensional floor plans I used a Python module called meshcut.py
(kindly provided by Yimeng Li.) Figure \ref{fig:Allensville_3D} shows
a 3D representation of the Allensville apartment. When a 2D images is
extracted at a height of 0.5 meters the result is found in
\ref{fig:Allensville_2D}.  This picture is an image saved from the
Python module matplotlib. But it can be converted to an
OpenCV-compatible numpy array and mantipulated in memory or saved to
file.

\begin{centering}
\begin{figure}[ht]
\caption{3D Image of Allensville}
\centering
\includegraphics[width=7cm]{Allensville_thumb.png}
\label{fig:Allensville_3D}
\end{figure}
\begin{figure}[ht]
\caption{2D Image of Allensville at 0.5 m}
\centering
\includegraphics[width=7cm]{Allensville.png}
\label{fig:Allensville_2D}
\end{figure}
\end{centering}

\subsection{Preparing Floorplan for Geometry Checks}

The size of the buildings are not large compared to the expected size
of robot we would expect to operate in them. There also need to be
many checks of lines for collisions with solid objects (walls and
other items in the rooms.) An efficient way to do these checks is by
rasterizing the maps and using Bresenham's line-drawing algorithm to
draw on the image. Solid areas are represented as black pixels, and
free areas as white pixels. The way to do this to use opencv.floodFill()
starting at a point in the free space to fill the freespace with white
pixels.

Even though the dataset documentation say most of the Gibson maps have
hard the holes filled, it is still the case that the 2D has visible
holes in the walls. This makes the floodFill() operation fail. These
gaps can be filled ``manually'', but this is tedious. So a function
finds the end-points to all lines, and connects them to the nearest
endpoint of another line. This can be done efficiently with numpy
representations of the line arrays. Not all maps work, but enough to
have a decent set to work with.

Finally, There are still small gaps between objects that are evidently
too small for a robot to fit through, but which the RRT lines would
traverse. To get rid of these, opencv.erode is applied to the
image. This also has the effect of providing some space from the walls
that represents the thickness of a robot. Frankly, I used only a few
pixels of erosion, so the spacing effect is not realistically
large. However, more erosion could certainly be used to get the
correct effect.

Figure \ref{fig:free} shows a rasterized free space map for Allensville after
gaps have been filled and erosion applied.

\begin{centering}
\begin{figure}[ht]
\caption{Allensville free space image.}
\centering
\includegraphics[width=7cm]{free.png}
\label{fig:free}
\end{figure}
\end{centering}

In summary, the algorithm to convert a set of numpy arrays, each
representing a line in the map, to the rasterized free space image,
is:

\begin{enumerate}
\item Find minimum and maximum x and y positions of lines.
\item Pad lines out with configurable quanity (typically 0.5 m in examples.)
\item Draw lines with opencv.polylines() on free image initialized to black.
\item Calculate lines to fill gaps and draw free image.
\item Floodfill image with white starting at fixed point in image (this should be made configurable.)
\item Erode image with a 5x5 mask a configurable number of times (5 by default.)
\end{enumerate}

\subsection{Creating the RRT}

The algorithm to produce the RRT is fairly simple:

\begin{enumerate}
\item Find random starting point in free space.
\item Perform $N$ times.
\begin{enumerate}
\item Find random point $s$ in anywhere the image.
\item Find existing point $t$ in tree closest to $s$.
\item If line of length $d$ can be drawn from $s$ to $t$ without
  hitting an obstruction (or, optionally, another line), then draw it,
  and add end of line as a new point in the RRT.
\end{enumerate}
\end{enumerate}

The check for an obstruction is made by using Bresenham line-drawing
to trace the line on the free-space image. If it hits a black pixel
then the line is not used. If the option not to cross tree lines is
specified, then a copy of the free space image is made, and each
successful new edge is drawn in black on it. This prevents future
edges from crossing it.  A sample RRT is shown in
\ref{fig:floormap}. The number of nodes to draw in the tree is
configurable. For Allensville 5000 is sufficient to fill the
graph. Larger spaces require more.

\begin{centering}
\begin{figure}[ht]
\caption{Allensville RRT} \centering
\includegraphics[width=7cm]{Allensville_floormap.png}
\label{fig:floormap}
\end{figure}
\end{centering}

Two programs can be used to produce an RRT: make\_rrt.py, and
rrtgui.py. Both use the library file rrt.py. make\_rrt.py creates RRTs
with several command line options and saves the floormap image, with
the RRT, the free image, and the parameters to files. rrtgui.py is a
demonstration program that displays an animation of the RRT as rrt.py
creates it. I did not get to the point of saving the products of
rrtgui.py to file.

Before evaluating the usefulness of the RRT, we'll examine the method
to find paths between arbitrary points.

\subsection{Finding a Path Using A* Search}

Given a starting point, and an ending point, both in real world
coordinate measured in meters, A* is used to find a path on the tree
between them. This algorithm is basically a direct implementation of
the one found in Russell and Norvig. The only difficulty was finding a
heapq with replacement. I didn't find one, but subclassed the standard
Python module as AStarHeap to work with special objects that have a
deleted flag to support deletion. The AStarHeap has a dictionary of
nodes to track when a node value already exists. This allows it to
implement a method to see if a nodes already exists with a worse cost.

So the algorithm uses the free image to find the closest node to each
of the start and end points, and then find the path. This is very
fast. A sample path is shown in \ref{fig:path_1}. The path start at
the green circle and ends at the red one. This path looks reasonable,
but because it has to follow the meandering RRT path, it doesn't
really look as good as it could be. 

\begin{centering}
\begin{figure}[ht]
\caption{Allensville Sample Path 1} \centering
\includegraphics[width=7cm]{Allensville_floormap_with_path_1.png}
\label{fig:path_1}
\end{figure}
\end{centering}

If we look carefully at the circled area in \ref{fig:gap} we see that
the leaves of the RRT do not meet up behind the object (a sofa.) This
is because an RRT really is a tree, not a graph. This leads to a path
from one side of the sofa to the other that goes around in front of
is, as shown in \ref{fig:path_2}.

\begin{centering}
\begin{figure}[ht]
\caption{Allensville RRT Gap} \centering
\includegraphics[width=7cm]{Allensville_RRT_Gap.png}
\label{fig:gap}
\end{figure}
\end{centering}

\begin{centering}
\begin{figure}[ht]
\caption{Allensville Sample Path 2} \centering
\includegraphics[width=7cm]{Allensville_gap_path.png}
\label{fig:path_2}
\end{figure}
\end{centering}

\subsection{Refining the Path}

In order to make the path more efficient we do the following:

\begin{enumerate}
\item Create a new graph with just the nodes and edges of the path.
\item Repeatedly add random edges between pairs of unconnected nodes.
\item Check cost of A* search for graph (since it is no longer a tree.)
\item When the cost appears to converge (no more improvements are
  being made to the cost of A*) then return the new path.
\end{enumerate}

We can see the effect in \ref{fig:path_3} of refining the path from
\ref{fig:path_2}. The path now goes directly behind the sofa, instead
of going around the front. But rather than refine the path every time,
we'd rather be able to have a static map and calculate the path one time.

\begin{centering}
\begin{figure}[ht]
\caption{Refined Path} \centering
\includegraphics[width=7cm]{Allensville_path_3.png}
\label{fig:path_3}
\end{figure}
\end{centering}

\subsection{Refining the Map}

To make a fixed map that is more efficient, we perform the the steps
in the previous section a configurable number of times on random start
and end points. After each iteration, we add the new path the the map
graph. Once this is done we have a graph where the most common paths
between major areas of the building have more direct straight-line
routes.

Figure \ref{fig:refined_map} shows a map with 100 added paths. Figures
\ref{fig:path_4} and \ref{fig:path_5} show paths created using the
refined map.

\begin{centering}
\begin{figure}[ht]
\caption{Refined Map} \centering
\includegraphics[width=7cm]{Allensville_refined_map.png}
\label{fig:refined_map}
\end{figure}
\end{centering}

\begin{centering}
\begin{figure}[ht]
\caption{Refined Path 2} \centering
\includegraphics[width=7cm]{Allensville_better_1.png}
\label{fig:path_4}
\end{figure}
\end{centering}

\begin{centering}
\begin{figure}[ht]
\caption{Refined Path 3} \centering
\includegraphics[width=7cm]{Allensville_better_2.png}
\label{fig:path_5}
\end{figure}
\end{centering}

\section{Results}

Several results have already been shown. To demonstrate
the fairly general applicability of this approach, here are
some refined maps of other Gibson environments.

\subsection{References}

List and number all bibliographical references in 9-point Times,
single-spaced, at the end of your paper. When referenced in the text,
enclose the citation number in square brackets, for
example~\cite{Authors06}.  Where appropriate, include the name(s) of
editors of referenced books.

%-------------------------------------------------------------------------
\subsection{Illustrations, graphs, and photographs}

All graphics should be centered.  Please ensure that any point you wish to
make is resolvable in a printed copy of the paper.  Resize fonts in figures
to match the font in the body text, and choose line widths which render
effectively in print.  Many readers (and reviewers), even of an electronic
copy, will choose to print your paper in order to read it.  You cannot
insist that they do otherwise, and therefore must not assume that they can
zoom in to see tiny details on a graphic.

When placing figures in \LaTeX, it's almost always best to use
\verb+\includegraphics+, and to specify the  figure width as a multiple of
the line width as in the example below
{\small\begin{verbatim}
   \usepackage[dvips]{graphicx} ...
   \includegraphics[width=0.8\linewidth]
                   {myfile.eps}
\end{verbatim}
}


%-------------------------------------------------------------------------
\subsection{Color}

Color is valuable, and will be visible to readers of the electronic copy.
However ensure that, when printed on a monochrome printer, no important
information is lost by the conversion to grayscale.

%------------------------------------------------------------------------
\section{Final copy}

You must include your signed IEEE copyright release form when you submit
your finished paper. We MUST have this form before your paper can be
published in the proceedings.

Please direct any questions to the production editor in charge of these
proceedings at the IEEE Computer Society Press: Phone (714) 821-8380, or
Fax (714) 761-1784.

{\small
\bibliographystyle{ieee}
\bibliography{egbib}
}

\end{document}
