\documentclass[10pt,twocolumn,letterpaper]{article}

\usepackage{cvpr}
\usepackage{times}
\usepackage{epsfig}
\usepackage{graphicx}
\usepackage{amsmath}
\usepackage{amssymb}

% Include other packages here, before hyperref.

% If you comment hyperref and then uncomment it, you should delete
% egpaper.aux before re-running latex.  (Or just hit 'q' on the first latex
% run, let it finish, and you should be clear).
\usepackage[pagebackref=true,breaklinks=true,letterpaper=true,colorlinks,bookmarks=false]{hyperref}

\cvprfinalcopy % *** Uncomment this line for the final submission

\def\cvprPaperID{****} % *** Enter the CVPR Paper ID here
\def\httilde{\mbox{\tt\raisebox{-.5ex}{\symbol{126}}}}

% Pages are numbered in submission mode, and unnumbered in camera-ready
\ifcvprfinal\pagestyle{empty}\fi
\begin{document}

%%%%%%%%% TITLE
\title{Using RRTs to Navigate Gibson Environments}

\author{Paul McKerley\\
George Mason University\\
{\tt\small pmckerle@masonlive.gmu.edu}
}

\frenchspacing
\maketitle
\thispagestyle{empty}

%%%%%%%%% ABSTRACT
\begin{abstract}
  Produce efficient maps to navigate two-dimensional floorplans
  extracted from three-dimensional Gibson Environment buildings.
\end{abstract}

%%%%%%%%% BODY TEXT
\section{Introduction}

  The goal of this project was to produce road maps for Gibson
  Environment buildings that would allow robots to navigate from one
  arbitrary point in the map to another. This navigation should find
  reasonable straight-line paths. My approach was to use
  Rapidly-Exploring Random Trees, developed by
  Lavalle and Kuffner~\cite{lavalle01}, which provided maps in a matter of
  seconds. A* search can traverse these maps, and the results paths
  refined to be straighter. Many samples paths can be found and
  refined, with the resulting paths added to the map. These refined
  paths give highly efficient paths between rooms.

\section{Approach}

There are several steps requirement to produce a roadmap of a Gibson
Environment structure.

\subsection{Extract Two-Dimensional Floorplan}

The Gibson buildings come as three-dimensional graphs from the
GibsonEnv database~\cite{xiazamirhe2018gibsonenv}. To extract
two-dimensional floor plans I used a Python module called
meshcut.py~\cite{jrebetez00} (kindly provided by Yimeng Li.) Figure
\ref{fig:Allensville_3D} shows a 3D representation of the Allensville
apartment. When a 2D images is extracted at a height of 0.5 meters the
result is found in \ref{fig:Allensville_2D}.  This picture is an image
saved from the Python module matplotlib. But it can be converted to an
OpenCV-compatible numpy array and mantipulated in memory or saved to
file.

\begin{centering}
\begin{figure}[ht]
\caption{3D Image of Allensville}
\centering
\includegraphics[width=7cm]{Allensville_thumb.png}
\label{fig:Allensville_3D}
\end{figure}
\begin{figure}[ht]
\caption{2D Image of Allensville at 0.5 m}
\centering
\includegraphics[width=7cm]{Allensville.png}
\label{fig:Allensville_2D}
\end{figure}
\end{centering}

\subsection{Preparing Floorplan for Geometry Checks}

The size of the buildings are not large compared to the size of robot
we would expect to operate in them. There also need to be many checks
of lines for collisions with solid objects (walls and other items in
the rooms.) An efficient way to do these checks is by rasterizing the
maps and using Bresenham's line-drawing algorithm to find obstructions
on the image. Solid areas are represented as black pixels, and free
areas as white pixels. To do this I used use opencv.floodFill()
starting at a point in the free space to fill the freespace with white
pixels.

Even though the dataset documentation say most of the Gibson maps have
the holes filled, it is still the case that the 2D cross-sections have
visible holes in the walls. This makes the floodFill() operation
fail. These gaps can be filled ``manually'', by examining the
cross-sections and adding appropriate lines, but this is tedious. So a
function finds the end-points to all lines, and connects them to the
nearest endpoint of another line. This can be done efficiently with a
numpy representations of the line arrays. Not all maps work, but
enough to have a decent set to work with.

Finally, There are still small gaps between objects that are evidently
too small for a robot to fit through, but which the RRT lines would
traverse. To get rid of these, opencv.erode is applied to the
image. This also has the effect of providing some space from the walls
that represents the thickness of a robot. Frankly, I used only a few
pixels of erosion, so the spacing effect is not realistically
large. However, more erosion could certainly be used to get the
correct effect.

Figure \ref{fig:free} shows a rasterized free space map for Allensville after
gaps have been filled and erosion applied.

\begin{centering}
\begin{figure}[ht]
\caption{Allensville free space image.}
\centering
\includegraphics[width=7cm]{free.png}
\label{fig:free}
\end{figure}
\end{centering}

In summary, the algorithm to convert a set of numpy arrays, each
representing a line in the map, to the rasterized free space image,
is:

\begin{enumerate}
\item Find minimum and maximum x and y positions of lines.
\item Pad lines out with configurable quanity (typically 0.5 m in examples.)
\item Draw lines with opencv.polylines() on free image initialized to black.
\item Calculate lines to fill gaps and draw free image.
\item Floodfill image with white starting at a fixed point in image (this should be made configurable.)
\item Erode image with a 5x5 mask and a configurable number of times
  (5 by default.)
\end{enumerate}

\subsection{Creating the RRT}

The algorithm to produce the RRT is fairly simple:

\begin{enumerate}
\item Find random starting point in free space.
\item Perform $N$ times.
\begin{enumerate}
\item Find random point $s$ in anywhere the image.
\item Find existing point $t$ in tree closest to $s$.
\item If line of length $d$ can be drawn from $s$ to $t$ without
  hitting an obstruction (or, optionally, another line), then draw it,
  and add end of line as a new point in the RRT.
\end{enumerate}
\end{enumerate}

The check for an obstruction is made by using Bresenham line-drawing
to trace the line on the free-space image. If it hits a black pixel
then the line is not used. If the option not to cross tree lines is
specified, then a copy of the free space image is made, and each
successful new edge is drawn in black on it. This prevents future
edges from crossing it.  A sample RRT is shown in
\ref{fig:floormap}. The number of nodes to draw in the tree is
configurable. For Allensville 5000 is sufficient to fill the
graph. Larger spaces require more nodes.

\begin{centering}
\begin{figure}[ht]
\caption{Allensville RRT} \centering
\includegraphics[width=7cm]{Allensville_floormap.png}
\label{fig:floormap}
\end{figure}
\end{centering}

Two programs can be used to produce an RRT: make\_rrt.py, and
rrtgui.py. Both use the library file rrt.py. make\_rrt.py creates RRTs
with several command line options and saves the floormap image, with
the RRT, the free image, and the parameters to files. rrtgui.py is a
demonstration program that displays an animation of the RRT as rrt.py
creates it. I did not get to the point of saving the products of
rrtgui.py to file.

Before evaluating the usefulness of the RRT, we'll examine the method
to find paths between arbitrary points.

\subsection{Finding a Path Using A* Search}

Given a starting point, and an ending point, both in real world
coordinate measured in meters, A* is used to find a path on the tree
between them. This algorithm is basically a direct implementation of
the one found in Russell and
Norvig~\cite{Russell:2009:AIM:1671238}. The only difficulty was
finding a heapq with replacement. I didn't find one, but subclassed
the standard Python module as AStarHeap to work with special objects
that have a deleted flag to support replacement. The AStarHeap has a
dictionary of nodes to track when a node value already exists. This
allows it to implement a method to see if a node already exists with
a worse cost.

The algorithm uses the free image to find the closest node to each
of the start and end points, and then find the path. This is very
fast. A sample path is shown in \ref{fig:path_1}. The path starts at
the green circle and ends at the red one. This path looks reasonable,
but because it has to follow the meandering RRT path, it doesn't
really look as good as it could be. 

\begin{centering}
\begin{figure}[ht]
\caption{Allensville Sample Path 1} \centering
\includegraphics[width=7cm]{Allensville_floormap_with_path_1.png}
\label{fig:path_1}
\end{figure}
\end{centering}

There are other inefficiencies besides the meandering RRT path. If we
look carefully at the circled area in \ref{fig:gap} we see that the
leaves of the RRT do not meet up behind the object (a sofa.) This is
because an RRT really is a tree, not a graph, so the edges never
connect to existing nodes. This leads to a path from one side of the
sofa to the other that goes around in front of is, as shown in
\ref{fig:path_2}.

\begin{centering}
\begin{figure}[ht]
\caption{Allensville RRT Gap} \centering
\includegraphics[width=7cm]{Allensville_RRT_Gap.png}
\label{fig:gap}
\end{figure}
\end{centering}

\begin{centering}
\begin{figure}[ht]
\caption{Allensville Sample Path 2} \centering
\includegraphics[width=7cm]{Allensville_gap_path.png}
\label{fig:path_2}
\end{figure}
\end{centering}

\subsection{Refining the Path}

In order to make the path more efficient we do the following:

\begin{enumerate}
\item Create a new graph with just the nodes and edges of the path.
\item Repeatedly add random edges between pairs of unconnected nodes.
\item Check cost of A* search for new graph (since it is no longer a tree.)
\item When the cost appears to converge (no more improvements are
  being made to the cost of A*) then return the new path.
\end{enumerate}

We can see the effect in \ref{fig:path_3} of refining the path from
\ref{fig:path_2}. The path now goes directly behind the sofa, instead
of going around the front. But rather than refine the path every time,
we'd rather be able to have a static map and calculate the path one time.

\begin{centering}
\begin{figure}[ht]
\caption{Refined Path} \centering
\includegraphics[width=7cm]{Allensville_path_3.png}
\label{fig:path_3}
\end{figure}
\end{centering}

\subsection{Refining the Map}

To make a fixed map that is more efficient, we perform the the steps
in the previous section a configurable number of times on random start
and end points. After each iteration, we add the new path the the map
graph. Once this is done we have a graph where the most common paths
between major areas of the building have more direct straight-line
routes.

Figure \ref{fig:refined_map} shows a map with 100 added paths. Figures
\ref{fig:path_4} and \ref{fig:path_5} show paths created using the
refined map.

\begin{centering}
\begin{figure}[ht]
\caption{Refined Map} \centering
\includegraphics[width=7cm]{Allensville_refined_map.png}
\label{fig:refined_map}
\end{figure}
\end{centering}

\begin{centering}
\begin{figure}[ht]
\caption{Refined Path 2} \centering
\includegraphics[width=7cm]{Allensville_better_1.png}
\label{fig:path_4}
\end{figure}
\end{centering}

\begin{centering}
\begin{figure}[ht]
\caption{Refined Path 3} \centering
\includegraphics[width=7cm]{Allensville_better_2.png}
\label{fig:path_5}
\end{figure}
\end{centering}

\section{Results}

Several results have already been shown. To demonstrate the
scalability of this approach, figure \ref{fig:brinnon_map} shows a map
for the Brinnon environment. Brinnon is a much larger environment than
Allensville. The latter is about 8m by 8m, and has about seven
rooms. Brinnon is 20m by 25m, and has about 14 seperate rooms. Fewer
larger environments can be made free of holes, and this is the largest
one I found. It took approximately 60 seconds to create the 2D view of
the Brinnon house; 23 seconds to create the RRT; and another 130
seconds to create the refined map. Once these are done, creating paths
is nearly instantaneous. Figure \ref{fig:brinnon_path_1} shows one
such path. If this were being used by a robot in this environment, the
time to create the map would be small, and subsequent searches for
efficient paths would be negligable.

\begin{centering}
\begin{figure}[ht]
\caption{Brinnon Map} \centering
\includegraphics[width=7cm]{Brinnon_refined.png}
\label{fig:brinnon_map}
\end{figure}
\end{centering}

\begin{centering}
\begin{figure}[ht]
\caption{Brinnon Path} \centering
\includegraphics[width=7cm]{Brinnon_path_1.png}
\label{fig:brinnon_path_1}
\end{figure}
\end{centering}

I admit I didn't spent much time comparing my approach to other
techniques. I spent most of my time working on my program and
improving it. I wrote the graphical program because I thought the
animation would be pretty neat, and it would be easier to test various
environments with it. These things were true, but writing the UI took
quite a bit of time because I've never done much graphical programming
in Python. 

\section{Conclusion}

Using RRTs proved a promising approach for creating static maps for
navigating Gibson environment. In a matter of a few minutes detailed
maps based on RRTs can be produced that will allow paths to be
generated using A* search.

One limitation of the approach is that the refining relies on
line-of-sight between vertexes. In these fairly simple environments
this does not appear to be a problem. If there were complex,
zig-zagging corriders, however, this limitations might be more
apparent in the results.

There are command line programs (make\_rrt.py and find\_path.py) which
have many command-line parameters available to control the creation of
the maps. They also save the generated maps and paths to files. The
GUI tool, rrtgui.py, has only hard-coded values (except for choosing
the environment), and it cannot save the maps and paths it
produces. Further work would be to enhance rrtgui.py to allow setting
of parameters and saving work.py. Also, the command-line tools do not
do the refining. They would need to have these capabilities added to
be useful.

The code I wrote is available in a public
\href{https://github.com/tenther/cs685-project}{Github repository}. The primary code artifacts are:

\begin{enumerate}
\item rrt.py: Library of functions for extracting a cross-section from
  a 3D represenation of a building, creating the free-space image,
  generating RRTs, and finding paths with A* search.
\item bresenham.py: My implementation of the Bresenham line-drawing algorithm.
\item find\_path.py: Command-line program to find a path given start and end points and
  an RRT file. 
\item make\_rrt.py: Command-line program use rrt.py to create an RRT
  from a Gibson Environment building.
\item rrtgui.py: A graphical program to read Gibson Environment files,
  create RRTs, find paths, refine paths, and perform multi-path refining.
\end{enumerate}

There are some other Python programs in there, but they are mostly cruft.

\bibliography{CS685_McKerley_Project}
\bibliographystyle{ieee}

\end{document}
